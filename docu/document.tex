\documentclass{article}
\usepackage[utf8]{inputenc}
\usepackage{graphicx}
\usepackage{tikz}
\usepackage{float}
\usepackage{textcomp}

\usetikzlibrary{positioning,fit,calc,arrows.meta, shapes}
\graphicspath{ {images/} }

%Tot això hauria d'anar en un pkg, però no sé com és fa
\newcommand*{\assignatura}[1]{\gdef\1assignatura{#1}}
\newcommand*{\grup}[1]{\gdef\3grup{#1}}
\newcommand*{\professorat}[1]{\gdef\4professorat{#1}}
\renewcommand{\title}[1]{\gdef\5title{#1}}
\renewcommand{\author}[1]{\gdef\6author{#1}}
\renewcommand{\date}[1]{\gdef\7date{#1}}
\renewcommand{\maketitle}{ %fa el maketitle de nou
    \begin{titlepage}
        \raggedright{UNIVERSITAT DE LLEIDA \\
            Escola Politècnica Superior \\
            Grau en Enginyeria Informàtica\\
            \1assignatura\\}
            \vspace{5cm}
            \centering\huge{\5title \\}
            \vspace{3cm}
            \large{\6author} \\
            \normalsize{\3grup}
            \vfill
            Professorat : \4professorat \\
            Data : \7date
\end{titlepage}}
%Emplenar a partir d'aquí per a fer el títol : no se com es fa el package
%S'han de renombrar totes, inclús date, si un camp es deixa en blanc no apareix

\tikzset{
	%Style of nodes. Si poses aquí un estil es pot reutilitzar més facilment
	pag/.style = {circle, draw=black,
                           minimum width=0.75cm, font=\ttfamily,
                           text centered}
}
\renewcommand{\figurename}{Figura}
\title{Segona pràctica d'Intel·ligència Artificial}
\author{Ian Palacín Aliana i Joaquim Picó Mora}
\date{Dimecres 11 de Desembre}
\assignatura{Inteligència Artificial}
\professorat{Eduard Torres}
\grup{}

%Comença el document
\begin{document}
\maketitle
\thispagestyle{empty}

\newpage
\pagenumbering{roman}
\tableofcontents
\newpage
\pagenumbering{arabic}
%
\section{Modelización de problemas de grafos}
\subsection{Maximum Clique}
Donat un graf, es genera per mitjà dels seus vèrtexs el graf complet. Un cop generat se n'esborren les arestes del graf original obtenint així les arestes complementàries. Finalment aquestes seran les que afegirem com a clàusules a la fórmula wcnf:
\\\\
(n1,n2)=(n1 v n2) 
\subsection{Max-Cut}
Per cada aresta es crearà dos clàusules soft amb weight = 1 tal que:
\\\\
(n1,n2)=(n1 v n2) i (-n1 v -n2)
\\\\
D'aquesta forma quan un dels vèrtexs és assignat a verdader implica que l'altre ha de ser fals.
\section{Transformación (1,3) Weighted Partial MaxSAT}
Per a cada clàusula soft es crearà una nova clàusula la qual contindrà un únic literal també nou. A l'hora de crear les clàusules hard, es tindran en compte les següents situacions:\\\\
-Que la llargada de la clàusula sigui exactament 3: En aquest cas s'afegirà la clàusula directament a la fórmula.\\\\
-Que la llargada de la clàusula sigui menor que 3: En aquest cas es multiplicarà l'array per 3 i s'agafaran els tres primers elements.\\\\
-Que la llargada de la clàusula sigui major que 3: En aquest cas, es dividirà la clàusula en diferents clàusules que es solaparan per mitjà de noves variables al final i a l'inici de les noves clàusules provinents de l'original.
Per exemple: (x1 v x2 v x3 v x4) = (x1 v x2 v x5) i (-x5 v x3 v x4) 

\section{Modelització de Software Package Upgrades}
En el que respecta a la modelització de Software Package Upgrades
hi havia dos tasques principals. La primera consistia en tractar
la instància d'entrada per tal de treballar amb les dades. La 
segona tasca era la de la modelització com a tal.\\
Tractar el fitxer d'entrada ha estat trivial, ja que python
facilita molt el treball, símplement, \textbf{n} per definir
variables, \textbf{d} per dependències i \textbf{c} per 
conflictes.\\
Una vegada tractades les dades, amb \textbf{wcnf} i msat\_runner
modelar el problema ha estat més fàcil.\\
 En primer lloc, el 
model serà de \textbf{wpms}, per tant tindrem clàusules hard i
clàusules soft. En el nostre cas, les clàusules soft seràn les
clàusules que solament contindràn les variables. Les clàusules
hard seràn aquelles que representin les dependències i els
conflictes.\\
Com hem dit, hi ha les clàusules que només
contindràn les variables. Les dependències són de la forma
pkg1 pkg2 pkg3 ..., on pkg1 té dependències sobre pkg2 o 
pkg3. Això es pot representar de la forma pkg1\textrightarrow(pkg2 o pkg3),
que en forma normal conjuntiva quedaria (!pkg1 v pkg2 v pkg3).
Una vegada en CNF, afegir la clàusula a la fòrmula amb wcnf és
trivial. Per últim, per modelar els conflictes, que són del tipus pkg1 pkg2,
on pkg1 té conflicte amb pkg2 és símplement (!pkg1 v !pkg2).\\
Una vegada s'ha cridat al MaxSAT amb la fòrmula anterior, i aquest
ens ha passat el resultat, com que volem saber quins paquets
no s'han pogut instal·lar, és a dir, quines variables no s'han
pogut satisfer, filtrem els paquets que siguin positius i ens
quedem amb els negatius.\\
Arribats a aquest punt solament queda imprimir-ho per sortida estàndard
juntament amb el nombre de variables no satisfetes.
\end{document}











