\documentclass{article}
\usepackage[utf8]{inputenc}
\usepackage{graphicx}
\usepackage{tikz}
\usepackage{float}
\usepackage{textcomp}

\usetikzlibrary{positioning,fit,calc,arrows.meta, shapes}
\graphicspath{ {images/} }

%Tot això hauria d'anar en un pkg, però no sé com és fa
\newcommand*{\assignatura}[1]{\gdef\1assignatura{#1}}
\newcommand*{\grup}[1]{\gdef\3grup{#1}}
\newcommand*{\professorat}[1]{\gdef\4professorat{#1}}
\renewcommand{\title}[1]{\gdef\5title{#1}}
\renewcommand{\author}[1]{\gdef\6author{#1}}
\renewcommand{\date}[1]{\gdef\7date{#1}}
\renewcommand{\maketitle}{ %fa el maketitle de nou
    \begin{titlepage}
        \raggedright{UNIVERSITAT DE LLEIDA \\
            Escola Politècnica Superior \\
            Grau en Enginyeria Informàtica\\
            \1assignatura\\}
            \vspace{5cm}
            \centering\huge{\5title \\}
            \vspace{3cm}
            \large{\6author} \\
            \normalsize{\3grup}
            \vfill
            Professorat : \4professorat \\
            Data : \7date
\end{titlepage}}
%Emplenar a partir d'aquí per a fer el títol : no se com es fa el package
%S'han de renombrar totes, inclús date, si un camp es deixa en blanc no apareix

\tikzset{
	%Style of nodes. Si poses aquí un estil es pot reutilitzar més facilment
	pag/.style = {circle, draw=black,
                           minimum width=0.75cm, font=\ttfamily,
                           text centered}
}
\renewcommand{\figurename}{Figura}
\title{Tercera pràctica d'Intel·ligència Artificial}
\author{Ian Palacín Aliana i Joaquim Picó Mora}
\date{Diumenge 12 de Gener}
\assignatura{Inteligència Artificial}
\professorat{Jesús Ojeda}
\grup{}

%Comença el document
\begin{document}
\maketitle
\thispagestyle{empty}

\newpage
\pagenumbering{roman}
\tableofcontents
\newpage
\pagenumbering{arabic}
%
\section{T10 - Construcción del árbol de forma iterativa}
%
Este método, igual que el anterior, crea un arbol de decisión pero
esta vez se tiene que construir de forma iterativa. Para hacer-lo, se
ha implementado con un bucle que trabaja siempre que haya elementos en 
una lista. Esta lista será una pila donde cada partición realizada
se le añadirá una estructura de datos contenida por el dataset despues
de la partición, el nodo responsable de la partición i si es la rama
true o false. El hecho de guardar en la estructura el dataset juntamente con 
el nodo padre i la rama servirá para poder relacionar a posteriori los nodos
que conformaran el arbol.\\
Dentro de este bucle se realiza lo mismo que en la función recursiva, se busca
el elemento con el cual se consiga el mayor gain al dividir el dataset i
se crea el nodo a partir de este elemento.\\
Este nodo se introducirá en una lista donde iremnos guardando todos los 
nodos de decision que se crean junto con el nodo del cual derivan 
i la rama sobre la cual trabajan. Esta será la lista desde la qual se 
relacionaran los nodos i se terminará de construir el arbol.
%
\section{T12 - Función de clasificación}
Classify se ha pensado como una función recursiva que partiendo del nodo raíz
del arbol se va moviendo por las ramas hasta llegar a un nodo hoja en función
a los valores que tenga el nuevo objeto a classificar. Cuando se llega a un nodo 
hojase retorna su resultado.
\section{T13/14 - Evaluación del árbol}
Se puede llevar a cabo la evaluación del arbol de dos formas diferentes.\\ 
Una de ellas es dividiendo el dataset de forma manual en dos fitxeros, training\_set.data i test\_set.data i la otra és usando la función split\_set, la qual divide el dataset en dos partes con los objetos distribuidos de forma random. Esta función recive como parametros un dataset i un porcentaje. Este último és el porcentaje que queremos que tenga training\_set respecto del data\_set al completo.\\
Si ejecutamos la función test performance con el traning\_set a 70\% i a 90\%, vemos que el porcentaje de aciertos de la ejecución con 90\% practicamente siempre és mayor. Esto se deve a que nuestra IA ha sido entrenada con mayor volumen de datos, i por lo tanto puede realizar una mejor classificación.
%
\end{document}







